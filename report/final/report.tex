\documentclass[11pt,a4paper]{article}
\usepackage[utf8]{inputenc}
\usepackage{amsmath}
\usepackage{amsfonts}
\usepackage{amssymb}
\usepackage{todonotes}
\usepackage{lscape}
\usepackage{cite}
\usepackage{hyperref}
\usepackage{afterpage}
\author{Pierre Gerard - Matteo Marra - Bruno Rocha Pereira}
\title{Given One \\ Intelligent Picture Browser \\ Requirements and Analysis report}
\date{December 19, 2016}

\begin{document}

\maketitle

\section{Introduction}

New generation user interfaces are developing day by day, with new tools, devices and different use cases showing up.

Of those devices, Virtual Reality sets are expanding as those devices get more affordable and many research team started working on it. 

Our brainstorming led to us imagining a future where we would use Virtual Reality on common basis, we thought about what an user would like to have and that doesn't exist yet, and how to implement it to make the user feel comfortable in this totally new environment. 

The system will offer the user a collection of pictures in a Virtual Reality environment, allowing him to browse them intelligently.

\section{Problem to be solved}

Since the numeric revolution, humans tend to take a ton of pictures. It is especially true during their holidays, usually coming back with thousands of pictures. The problem that arise then is that human don't usually have a mean to explore them all other than browsing through them one by one. So, the idea for the project is to create a \textit{Intelligent Picture Browser (IPB)} to fill that void.

\section{Requirement Analysis}

The product should browse the user complete gallery of pictures and select a subset of them. The subset selection will be based on a keyword/tag selection. The keyword/tag is automatically linked to pictures by a AI algorithm.

The user will be able to interact with the interface trough the virtual reality set and be able to select tag and some settings via voice.

\subsection{User characteristics}
The interface will be design to fit almost every individual having the capacity to use a virtual reality set and voice recognition. It will target technological novice as well as professional. Unfortunately, people with disabilities preventing them from using a VR set or voice recognition won't be able to use our system.

\subsection{Usability goals}

The usability of this project is going to be as much as straightforward as possible and will not require any particular skill or educational background. However having a regular access to technologies and computers will be a plus for a smoother use. 

\subsection{User experience goals}

The main goal is to make the user relive the moment immortalized by the pictures and helping him remember the event and feel the same emotion once again.

\section{Prototyping}

\subsection{Iteration 1}

%define problem/need and how our system is going to solve it and technologies used

\subsection{Iteration 2}

% define the requirements

\subsection{Iteration 3}

% getting a prototype where all technologies work separately (low-fidelity)

\subsection{Iteration 4}

% gettin a prototype where all techno work together

\subsection{Iteration 5}

% improvement to techno (or and and, image around, minors)

\subsection{Iteration 6}

% improvement (display of tag suggestion, pictures moving toward a farther, selecting pictures, moving upward and downward, minor)

% evaluation


\section{Intelligent Picture Browser}


% first clearly defined product wanted and requirement, iterative evolutionary prototyping with refinement PoC until an evaluted final product, we followed the init gant chart

% here insert iterative pictures of our system

\subsection{Interactions}

% voice recognition for tag and non-vive-command intuite interaction
% for intuitive interaction

\subsection{Architecture}

% NN for tag -> json

% Utinty interface <- json
% unity interface <- win10 speech recognizer


\subsection{Challenges}

% use unity and all its specificities

% poor software interoperability, as a result our project only work on win 10

% vr new and doc is bad

% require a monster computer

\subsection{Evaluation}

% usability testing -> ensure that user can use the product and they like it . 

% Student in CS and non-tech guy with no previous knwoledge on our product.

% Evalution take place in front of computer in natural setting

% summative evalution -> assess the final product. How well did we do ?


\end{document}




