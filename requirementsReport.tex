\documentclass[11pt,a4paper]{article}
\usepackage[utf8]{inputenc}
\usepackage{amsmath}
\usepackage{amsfonts}
\usepackage{amssymb}
\usepackage{todonotes}
\author{Pierre Gerard - Matteo Marra - Bruno Rocha Pereira}
\title{Given One \\ Intelligent Picture Browser \\ Requirements and Analysis report}
\begin{document}

\maketitle

\section{Introduction}
New generation user interfaces are developing day by day, with new tools, devices and different use cases showing up as the common users start using those devices everyday.
Of those devices, Virtual Reality is expanding as devices get more affordable and many research team started working on it. 

Imagining a future where we would use Virtual Reality on common basis, we thought about what an user could need that doesn't exist right now, and how to implement it to make the user feel comfortable in this totally new environment.

Imagine to have a collection of pictures you took on a trip, or on many trips. Imagine it now in a Virtual Reality environment, with pictures all over the place, like a 3D desk full of pictures not organized in album.

How can we use the Virtual Reality Environment and its computational power to make this collection of pictures organize and easily navigable?
The more user-friendly way is to have a system that recognizes your pictures, tagging them according to what is in the picture (e.g. a Dog, a sunset, ...), and allows you to query them via speech recognition. This query will filter in the environment the pictures you are interested in, and allow you to navigate through them.

This document includes the analysis of the requirements for the \textit{Intelligent Picture Browser} project, and is structured in the following way: Problem to be solved, Resolution requirements, Gantt chart and conclusions.

\section{Problem to be solved}

The aim of this project will be to solve a real life problem that we all experienced. Since the numeric revolution, humans tend to take a ton of pictures. It is especially the case during their holidays, usually coming back with thousands of pictures. The problem that arise then is that human don't have a mean to explore them all other than browsing through them one by one. So, the idea for the project is to create a \textit{Intelligent Picture Browser (IPB).}
 
\section{Resolution requirements}

The resolution of the problem will take advantage of new opportunities in human-computer interaction such as Virtual Reality and voice recognition. It will also take advantage of advance in the state of the art of image recognition using a neural network.

\section{Data requirements\todo{}}

\begin{itemize}
\item type of data
\item amount of data
\item data accuracy
\end{itemize}

\section{Environmental requirements (context of use)\todo{}}

\begin{itemize}
\item physical environment
- lighting, noise, movement, dust, ...
\item social environment
- synchronous or asynchronous sharing of data, co-located or distributed, ...
\item organisational environment
- user support, resources for training, how hierarchical is the management, ...
\item technical environment
- technologies the product will run on, compatibility, technological limitations, ...
\end{itemize}

\section{User characteristics\todo{}}



key attributes of intended user group
\begin{itemize}
\item abilities and skills
\item nationality and educational background
\item preferences
\item physical or mental disabilities
\item level of expertise (novice, expert, casual user, frequent user, ...)

\end{itemize}

user profile consists of a collection of attributes for a typical user


\section{Gantt chart}

\section{Conclusions}

\end{document}
